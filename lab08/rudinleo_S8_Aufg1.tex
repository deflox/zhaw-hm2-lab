\documentclass{article}

% PACKAGE DECLARATIONS

% unicode encoding
\usepackage[utf8]{inputenc}
% to set spacing of page margins
\usepackage[margin=1cm,nohead]{geometry}
% to include images
\usepackage{graphicx}
% to automatically create paragraphs when doing so in the code
\usepackage{parskip}
% math symbols and math features
\usepackage{amsmath}
\usepackage{amssymb}

% END PACKAGE DECLARATIONS

% disable page numbering
\pagenumbering{gobble}

\begin{document}

\section*{HM2 Serie 8 Aufgabe 1}
Leo Rudin

\subsection*{a)}
Ein einzelnes Teilstück eines Trapez wird folgendermassen berechnet: \(Tf = \frac{T(a) + f(b)}{2} \cdot (b-a)\)

Wenn man nun die einzelnen Stücke in die Formel einsetzt und aufsummiert erhält man automatisch die in der Aufgabe aufgeführt Formel:

\(Tf(h) = \sum_{i=0}^{n-1} \frac{y_i + y_{i+1}}{2} \cdot (x_{x+1} - x_i)\)

\subsection*{b)}
Eine Beispielsrechnung von \(y_1-y_6\)

Ein Abschnitt wird berechnet durch:

\(\frac{y_i + y_{i+1}}{2} \cdot h\)

Als Summe ergibt das:

\(\frac{y_1 + y_2}{2} \cdot h + \frac{y_2 + y_3}{2} \cdot h + \frac{y_3 + y_4}{2} \cdot h + \frac{y_4 + y_5}{2} \cdot h + \frac{y_5 + y_6}{2} \cdot h\)

Nun kann man \(\frac{1}{2} \cdot h\) ausklammern:

\((y1 + y2 + y2 + y3 + y3 + y4 + y4 + y5 + y5 + y6) \cdot \frac{1}{2} \cdot h\)

Aufteilung der Summen in Gruppen:

\([( y1 + y6 ) + (y2 + y2 + y3 + y3 + y4 + y4 + y5 + y5)] \cdot \frac{1}{2} \cdot h\)

Wenn man nun \(\frac{1}{2}\) wieder reinmultipliziert, fällt jeweils ein Teil der Paare weg:

\([\frac{y1+y2}{2} + (y2 + y3 + y4 + y5)] \cdot h\)

Nun kann der hintere Teil in eine Summe umgewandelt werden:

\([\frac{y1+y2}{2} + \sum_{i=1}^{n-1}(y2 + y3 + y4 + y5)] \cdot h\)

\end{document}
