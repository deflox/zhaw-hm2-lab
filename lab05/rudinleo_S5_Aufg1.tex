\documentclass{article}
\usepackage[utf8]{inputenc}

% PACKAGE DECLARATIONS

% unicode encoding
\usepackage[utf8]{inputenc}
% to set spacing of page margins
\usepackage[margin=1cm,nohead]{geometry}
% to include images
\usepackage{graphicx}
% to automatically create paragraphs when doing so in the code
\usepackage{parskip}
% math symbols and math features
\usepackage{amsmath}
\usepackage{amssymb}

% END PACKAGE DECLARATIONS

% disable page numbering
\pagenumbering{gobble}

\begin{document}

\section*{HM2 Serie 5 Aufgabe 1}
Leo Rudin

Datenpunkte:
\begin{itemize}
    \item \((x_0,y_0) = (4,6)\)
    \item \((x_1,y_1) = (6,3)\)
    \item \((x_2,y_2) = (8,9)\)
    \item \((x_3,y_3) = (10,0)\)
\end{itemize}

Es gibt insgesamt 4 Stützpunkte also ist n = 3, da der Algorithmus mit 0 beginnt.

Es wird nun mit i = 0,1,...,n-1 iteriert:

\textbf{1) Ausrechnen der \(a_i\) Werte:}
\begin{itemize}
    \item \(a_0 = y_0 = 6\)
    \item \(a_1 = y_1 = 3\)
    \item \(a_2 = y_2 = 9\)
\end{itemize}

\textbf{2) Ausrechnen der \(h_i\) Werte:}
\begin{itemize}
    \item \(h_0 = x_{0+1} - x_0 = 6 - 4 = 2\)
    \item \(h_1 = x_{1+1} - x_1 = 8 - 6 = 2\)
    \item \(h_2 = x_{2+1} - x_2 = 10 - 8 = 2\)
\end{itemize}

\textbf{3) Ausrechnen der \(c_i\) Werte:}

\(c_0 = 0\) \(c_3 = 0\)

Für \(c_1\) und \(c_2\) wird ein Gleichungssystem aufgestellt. Spezialfall, da nicht gilt: \(n \geq 4\) \(\rightarrow\) Es gibt nur genau zwei c-Werte zu berechnen:

\(2(h_0 + h_1)c_1 + h_1c_2 = 3\frac{y_2-y_1}{h_1} - 3\frac{y_1-y_0}{h_0} \rightarrow 8c_1 + 2c_2 = 3\frac{6}{2} - 3\frac{-3}{2} \rightarrow 8c_1 + 2c_2 = 13.5\)

\(h_1c_1 + 2(h_1 + h_2)c_2 = 3\frac{y_3-y_2}{h_2} - 3\frac{y_2-y_1}{h_1} \rightarrow 2c_1 + 8c_2 = 3\frac{-9}{2} - 3\frac{6}{2} \rightarrow 2c_1 + 8c_2 =  -22.5\)

In Matrixform:

\(
\begin{pmatrix}
8 & 2\\
2 & 8\\
\end{pmatrix}
\cdot
\begin{pmatrix}
c_1\\
c_2\\
\end{pmatrix}
=
\begin{pmatrix}
13.5\\
-22.5\\
\end{pmatrix}
\)

\(c_1 = 2.55\) \(c_2 = -3.45\)

\textbf{4) Ausrechnen der \(b_i\) Werte:}
\begin{itemize}
    \item \(b_0 = \frac{y_1-y_0}{h_0}-\frac{h_0}{3}(c_1+2c_0) = \frac{3-6}{2}-\frac{2}{3}(2.55) = -3.2\)
    \item \(b_1 = \frac{y_2-y_1}{h_1}-\frac{h_1}{3}(c_2+2c_1) = \frac{9-3}{2}-\frac{2}{3}(-3.45+2(2.55)) = 1.9\)
    \item \(b_2 = \frac{y_3-y_2}{h_2}-\frac{h_2}{3}(c_3+2c_2) = 
    \frac{0-9}{2}-\frac{2}{3}(2(-3.45)) = 0.1\)
\end{itemize}

\textbf{5. Ausrechnen der \(d_i\) Werte:}
\begin{itemize}
    \item \(d_0 = \frac{1}{3h_0}(c_1-c_0) = \frac{1}{6}(2.55-0) = 0.425\)
    \item \(d_1 = \frac{1}{3h_1}(c_2-c_1) = \frac{1}{6}(-3.45-2.55) = -1\)
    \item \(d_2 = \frac{1}{3h_2}(c_3-c_2) = \frac{1}{6}(0-(-3.45)) = 0.575\)
\end{itemize}

\hspace{5mm}

\(S_0(x) = 6 - 3.2(x-4) + 0.425(x-4)^3\)

\(S_1(x) = 3 - 1.9(x-6) + 2.55(x-6)^2 - 1(x-6)^3\)

\(S_2(x) = 9 + 0.1(x-8) -3.45(x-8)^2 + 0.575(x-8)^3\)

\end{document}
